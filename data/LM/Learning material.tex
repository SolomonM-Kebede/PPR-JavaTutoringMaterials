\documentclass[12pt,a4paper]{article}

\usepackage[utf8]{inputenc}
\usepackage[T1]{fontenc}
\usepackage{lmodern}
\usepackage{geometry}
\usepackage{hyperref}
\usepackage{setspace}

\geometry{margin=2.5cm}
\setstretch{1.2}

\title{Staff \& Salary Neo4j Project \\
\large Educational Java Project for Teaching Architecture, Factories, and Graph Databases}
\author{Prepared for Students (Group 8 PPR Tutorial): Prepared by Solomon}
\date{\today}

\begin{document}

\maketitle
\tableofcontents
\newpage

\section{Introduction}

This project is a minimal but realistic Java application designed to teach students:

\begin{itemize}
    \item how to structure a Java project cleanly,
    \item how to use a factory pattern to create domain objects,
    \item how to interact with Neo4j using nodes and relationships,
    \item how to separate model, persistence, factory, and REST layers,
    \item how to build a small web interface using Javalin, FreeMarker, and jQuery.
\end{itemize}

The domain is intentionally simple: employees (Staff) with associated salary information (Salary).  
Data is persisted in a Neo4j graph.

\section{Project Overview}

The project models:

\begin{itemize}
    \item \textbf{Staff} :an employee with ID, first name, last name, nickname.
    \item \textbf{Salary} : salary amount and currency.
\end{itemize}

Each staff node in Neo4j has:

\begin{itemize}
    \item properties (id, first name, last name, nickname)
    \item an outgoing relationship \texttt{HAS\_SALARY} to a Salary node
\end{itemize}

Students learn:

\begin{itemize}
    \item graph modeling,
    \item node creation,
    \item relationships,
    \item REST endpoints,
    \item template rendering (FTL),
    \item AJAX requests (jQuery).
\end{itemize}

\section{Package Structure}

The real project contains these core packages:

\begin{itemize}
    \item \textbf{database} :manages Neo4j connection
    \item \textbf{interfaces}: Staff and Salary domain interfaces
    \item \textbf{implementations} : Neo4j-based concrete implementations
    \item \textbf{factory} : StaffFactory and SalaryFactory (factory pattern)
    \item \textbf{rest}: REST routes using Javalin
    \item \textbf{templates} : FreeMarker templates includes jQuery(it can be separate .js file created)
\end{itemize}

This structure demonstrates clean separation of concerns.

\section{The \texttt{database} Package}

\subsection{Neo4jConnection}

This class wraps the embedded or file-based Neo4j database.

\textbf{Responsibilities:}
\begin{itemize}
    \item open the database
    \item expose a \texttt{GraphDatabaseService}
    \item ensure safe shutdown
\end{itemize}

\textbf{Important Concept for Students:}
A single connection object is passed into factories so they can create nodes and relationships.

\section{The \texttt{model} Package}

Contains the domain interfaces:

\subsection{Staff}

\begin{itemize}
    \item \texttt{String getId()}
    \item \texttt{String getFirstName()}
    \item \texttt{String getLastName()}
    \item \texttt{String getNickname()}
    \item \texttt{Salary getSalary()}
\end{itemize}

\subsection{Salary}

\begin{itemize}
    \item \texttt{double getSalary()}
    \item \texttt{String getCurrency()}
\end{itemize}

\section{The \texttt{impl} Package}

Concrete Neo4j-backed implementations.

\subsection{StaffNeo4jImpl}

Represents a single Staff node in Neo4j.

\textbf{Responsibilities:}

\begin{itemize}
    \item wrap a Neo4j \texttt{Node}
    \item provide getters for staff properties
    \item resolve linked Salary node
\end{itemize}

\subsection{SalaryNeo4jImpl}

Represents a Salary node in Neo4j.

\textbf{Responsibilities:}

\begin{itemize}
    \item wrap Neo4j node
    \item return salary properties
\end{itemize}

\section{The \texttt{factory} Package}

This is the architectural highlight of the project.

\subsection{StaffFactory}

\begin{itemize}
    \item takes a \texttt{Neo4jConnection} in the constructor
    \item can create new staff nodes
    \item can attach salary to staff
    \item can search by ID
    \item returns \texttt{StaffNeo4jImpl} objects
\end{itemize}

\textbf{Key teaching point:}  
Students learn dependency injection and factory responsibility.

\subsection{SalaryFactory}

Symmetric to StaffFactory, responsible for creating salary nodes.

\section{REST Layer (Javalin)}

Routes include:

\begin{itemize}
    \item \texttt{GET /staff} : renders staff table (FTL)
    \item \texttt{GET /staff/\{id\}} :renders single staff (FTL)
    \item \texttt{GET /api/staff} : returns JSON list of all staff
    \item \texttt{GET /api/staff/\{id\}} : returns single staff as JSON
\end{itemize}

\section{Frontend: FreeMarker + jQuery}

\subsection{FreeMarker Template (\texttt{staff.ftl})}

Contains:

\begin{itemize}
    \item staff table
    \item search bar
    \item reset button
\end{itemize}

\subsection{jQuery Search}

Example:

\begin{verbatim}
$.get("/api/staff/" + id, function(data) {
    // replace table content
});
\end{verbatim}

Students learn:

\begin{itemize}
    \item asynchronous requests
    \item updating HTML dynamically
\end{itemize}

\section{Learning Goals}

By completing this project, students learn:

\subsection{Backend Concepts}

\begin{itemize}
    \item What a model is
    \item What a factory is
    \item Clean separation of layers
    \item How graph databases work
    \item Relationship creation and traversal
\end{itemize}

\subsection{Frontend Concepts}

\begin{itemize}
    \item Templating with FreeMarker
    \item AJAX using jQuery
    \item Rendering dynamic content
\end{itemize}

\subsection{Architectural Skills}

\begin{itemize}
    \item correct use of packages
    \item avoiding mixing logic and UI
    \item writing clean, testable code
\end{itemize}

\section{Conclusion}

The project provides a complete mini-architecture using:

\begin{itemize}
    \item Java + Neo4j
    \item Factory Pattern
    \item Javalin REST API
    \item FreeMarker frontend
    \item AJAX search (jQuery)
\end{itemize}



\end{document}